\documentclass[11pt]{article}

\usepackage[margin=1in]{geometry}
\usepackage{amsmath,amsfonts,amssymb,amsthm}
\usepackage{braket}

\usepackage[dvipsnames]{xcolor}
\usepackage[colorlinks, linkcolor=OliveGreen, citecolor=purple,urlcolor=purple, bookmarks=true,backref=page]{hyperref}
\usepackage[capitalise, nameinlink]{cleveref}
\usepackage{physics}
% redefine formatting of backwards references
\renewcommand{\backref}[1]{}
\renewcommand{\backreftwosep}{, }
\renewcommand{\backreflastsep}{, }
\renewcommand{\backrefalt}[4]{%
\ifcase #1 %
%
\or
[p.\ #2]%
\else
[pp.\ #2]%
\fi}

% quantum commands
% \newcommand{\ketbra}[2]{|#1\rangle\langle#2|}
\newcommand{\ketbrasame}[1]{|#1\rangle\langle#1|}

% naming commands
\newcommand{\GHZ}{\mathrm{GHZ}}
\newcommand{\EPR}{\mathrm{EPR}}

\begin{document}

%%%%%%%%%%%%%%%%%%%%%%%%%%%%%%%%%%%%%%%%%%%%%%%%%%%%%%%%%%%%%%%%%%%%%%%%%%%%%%
\title{Project Proposal on Quantum Monte Carlo}

\author{Hannah Baek, Tobias Tian \\
University of British Columbia}

\date{\vspace{-2.5ex} October 20th, 2025}
\maketitle

%%%%%%%%%%%%%%%%%%%%%%%%%%%%%%%%%%%%%%%%%%%%%%%%%%%%%%%%%%%%%%%%%%%%%%%%%%%%%%
% \begin{abstract}
% This is a short description of the paper.
% \end{abstract}

%%%%%%%%%%%%%%%%%%%%%%%%%%%%%%%%%%%%%%%%%%%%%%%%%%%%%%%%%%%%%%%%%%%%%%%%%%%%%%
\section{Topic Description}
Quantum Monte Carlo (QMC) algorithms are classical computational methods used for simulating quantum systems without the need to access quantum hardware. These algorithms are often used to estimate important physical quantities, such as the ground state energy, in systems where Schr\"odinger's equations would be too complex to solve analytically or even numerically. \\
In this paper, we will explore the theoretical backgrounds and practical implications of three variants of QMC: Path Integral Monte Carlo (PIMC), Variational Monte Carlo (VMC), and Diffusion Monte Carlo (DMC). We will also provide sample code implementations that demonstrate the key ideas behind these algorithms.


%%%%%%%%%%%%%%%%%%%%%%%%%%%%%%%%%%%%%%%%%%%%%%%%%%%%%%%%%%%%%%%%%%%%%%%%%%%%%%
\section{Timeline}
Oct. 20 - Nov. 2: Start with understanding what each variant of QMC algorithm does. For a general overview, commence on an introductory article by Ceperly \cite{ceperly_10}. For PIMC, commence on the review article \cite{troyer_wiese_05}. For VMC and DMC, commence on the review article \cite{foulkes_mitas_needs_rajagopal_01}. \\
Nov. 3 - Nov. 16: Further study the mathematical and algorithmic setup for these computational methods, and write out the theoretical principles that supports the idea of QMC, by reading through lecture notes on this topic \cite{troyer_11}. Develop sample code that simulate physical systems, one potential example would be implementing Variational Monte Carlo to estimate ground state energy, one library that may assist with this process is QWalk \cite{wagner_bajdich_mitas_09}. \\
Nov. 17 - Nov. 30: Collect and visualize results from running simulations, and write the analysis and discussion surrounding the computational behaviour. \\
Dec. 1 - Dec. 7: Finalize the paper by integrating all sections. Polish writing, citations, and formatting. Refine sample code implementations and potential graphical figures. \\
The references used is a starting point for this project, more references will be added to the paper as this project progresses.

%%%%%%%%%%%%%%%%%%%%%%%%%%%%%%%%%%%%%%%%%%%%%%%%%%%%%%%%%%%%%%%%%%%%%%%%%%%%%%
% \section*{Acknowledgments}

% If you have discussions with your classmates about your chosen topic, you can thank them here.

%%%%%%%%%%%%%%%%%%%%%%%%%%%%%%%%%%%%%%%%%%%%%%%%%%%%%%%%%%%%%%%%%%%%%%%%%%%%%%
\bibliography{references}
\bibliographystyle{alphaurl}
\end{document}