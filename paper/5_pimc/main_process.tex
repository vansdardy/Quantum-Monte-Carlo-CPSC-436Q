\subsection{Main Process}
With the mathematical setup, the main process follows several steps as described below: \\
Step 1: Write the partition function in \cref{partition} in terms of a path integral like \cref{eq:path_integral}. \\
Step 2: For each particle of all $N$ particles, represent it in a ``world line'' such that $(\mathbf{r}_1^{i}, \mathbf{r}_2^i, \dots, \mathbf{r}_M^i)$ describes the path of particle $i$ over $M$ steps. On aggregate, this should give us a path of configurations as well $(\mathbf{R}_1, \mathbf{R}_2, \dots, \mathbf{R}_M)$. \\
Step 3: To obtain one path of configurations, each move is accepted or rejected based on the Metropolis algorithm similar to that of in \cref{metropolis} \cite{troyer_11}, and the displacement of configurations is being sampled from a Gaussian distribution similar to that in \cref{dmc_math}, which, in this case, is according to \cref{eq:transition_approx}. \\
Step 4: Obtain say $n$ paths. From these $n$ sample paths, it is possible to estimate a thermodynamic observable following a similar logic as estimating the ground-state energy:
$$\expval{\mathcal{O}} = \frac{1}{n}\Sum{i = 1}{n} \mathcal{O}(\{\mathbf{R}_i\})$$
where $\mathcal{O}(\{\mathbf{R}_i\})$ represents calculating the observable on the $i$th path sampled. The details of how to obtain the properties for a single path is not discussed here, as the mathematical details, once again, overshadow the key ideas of PIMC. \\
It is important, however, to highlight that PIMC suffers from the fermion sign problem as well, if not worse, compared to the DMC method. This is because DMC only evolves ``walkers'' (single points in a distribution), while PIMC evolves entire set of particles from one configuration to another.