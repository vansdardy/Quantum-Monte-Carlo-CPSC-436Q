\subsection{Mathematical Background}
All QMC methods start by investigating which Hamiltonian is of interest, in most cases, similar to what was mentioned in DMC, a general Hamiltonian that describes a many-body system takes the form of:
$$\hamiltonian = \hat{T} + \hat{V}$$
where $\hat{T}$ represents the kinetic term, and $\hat{V}$ represents the potential term, which may arise from different sources (one of such sources is the interaction between particles). \\
The key element that PIMC utilizes is the density matrix for the canonical ensemble which can be described as the following \cite{troyer_11} \cite{barker_79} \cite{kalos_whitlock_08}:
\begin{definition}
    The thermal density matrix $\exp(-\beta\hamiltonian)$ from which one can obtain thermal equilibrium takes the form where
    \begin{equation}\label{eq:thermal_density}
        \rho(\mathbf{R}, \mathbf{R}', \beta) = \mel{\mathbf{R}}{\exp(-\beta \hamiltonian)}{\mathbf{R}'}
    \end{equation}
    denotes the matrix elements, and $\beta$ is the inverse of the product of the Boltzmann constant and the temperature: $\beta = \frac{1}{k_B T}$
\end{definition}
\begin{definition}\label{partition}
    The partition function, a function of the thermodynamic state variables, in concern is described to be \cite{barker_79}
    \begin{equation}
        Z = \frac{1}{N!} \int_{\mathbf{R}} \rho(\mathbf{R}, \mathbf{R}, \beta) \dd \mathbf{R}
    \end{equation}
\end{definition}
By leveraging \cref{eq:thermal_density} and the product property of the thermal density matrix, the following stands \cite{troyer_11} \cite{barker_79} \cite{kalos_whitlock_08}
\begin{equation}\label{eq:transition}
    \rho(\mathbf{R}_1, \mathbf{R}_3, \beta_1 + \beta_2) = \int \rho(\mathbf{R}_1, \mathbf{R}_2, \beta_1)\rho(\mathbf{R}_2, \mathbf{R}_3, \beta_2) \dd \mathbf{R}_2
\end{equation}
By taking $M$ small time steps where $\tau = \frac{\beta}{M}$ as $M \to \infty$, it is possible to extend \cref{eq:transition} that \cite{troyer_11} \cite{barker_79}:
\begin{equation}\label{eq:path_integral}
    \rho(\mathbf{R}_1, \mathbf{R}_{M + 1}, \beta) = \int \int \cdots \int \rho(\mathbf{R}_1, \mathbf{R}_{2}, \tau) \rho(\mathbf{R}_2, \mathbf{R}_{3}, \tau) \times \cdots \times \rho(\mathbf{R}_{M}, \mathbf{R}_{M+1}, \tau) \dd \mathbf{R}_2 \mathbf{R}_3 \cdots \dd \mathbf{R}_M
\end{equation}
When $M$ is sufficiently large, or temperature $T$ is sufficiently high, it follows that $\tau$ must be small, which allows replacing unknown $\rho(\mathbf{R}_j, \mathbf{R}_{j+1}, \tau)$ with known functions through approximation \cite{troyer_11}. One such approximation echoes \cref{eq:green_approx} in DMC \cite{barker_79}, 
\begin{equation}\label{eq:transition_approx}
    \rho(\mathbf{R}, \mathbf{R}', \tau) \approx (2\pi\tau)^{-\frac{\nu N}{2}}\exp(-\frac{(\mathbf{R} - \mathbf{R}')^2}{2\tau} - \tau (\frac{V(\mathbf{R}) + V(\mathbf{R}')}{2}))
\end{equation}
in atomic units, where $\nu$ is the dimensionality of space (can take $\nu = 3$ for 3D space), and
$$(\mathbf{R} - \mathbf{R}')^2 = \Sumi{i}(\mathbf{r}_i - \mathbf{r}_i')^2$$
The Green's function in this case specifically refers to the part:
\begin{equation}
    G(\mathbf{R}, \mathbf{R}', \tau) = \exp(-\frac{(\mathbf{R} - \mathbf{R}')^2}{2\tau} - \tau (\frac{V(\mathbf{R}) + V(\mathbf{R}')}{2}))
\end{equation}
from \cref{eq:transition_approx}.