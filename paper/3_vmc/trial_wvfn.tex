\subsection{Selection and Optimization of the Trial Wavefunction}

In this section, we focus on zero-temperature VMC applied to electronic systems, where the 
wavefunction describes $N$ interacting electrons.  
Choosing an accurate trial wavefunction $\Psi_T$ is crucial: it determines not only the 
variational energy $E_V$ but also the statistical efficiency of the Monte Carlo sampling.  
The purpose of $\Psi_T$ is to provide the best possible approximation to the true ground state 
while remaining mathematically well-behaved and computationally tractable.

%---------------------------------------------------------
\subsubsection*{Motivation: Why Good Trial Wavefunctions Matter}

Although VMC is a powerful many-body method, its accuracy depends heavily on the quality of 
$\Psi_T$. Most ground-state calculations in electronic structure theory instead use 
less expensive methods such as Hartree--Fock (HF) or Kohn--Sham density functional theory 
(DFT). These methods scale favorably and are efficient to solve, but they have 
well-known failures: HF neglects correlation entirely (beyond exchange), while common DFT 
functionals struggle with strongly interacting systems, transition-metal atoms, 
van der Waals interactions, and dissociation regions. These limitations are further discussed in 
Ref.~\cite{FMNR01}, Sec.~II~C.
As a result, their predictive power is often limited.

Quantum Monte Carlo techniques successfully overcome many of these failures.  
Continuum QMC calculations can already achieve ``chemical accuracy''  
($\sim 1$ kcal/mol $\approx 0.004$ eV per molecule)  
for small systems and show no fundamental loss of accuracy for larger systems (Ref.~\cite{FMNR01}, Sec.~I).  
The computational cost of fermionic VMC and DMC scales approximately as $N^3$ with the number of 
electrons $N$, making simulations of moderately large systems feasible.  
Furthermore, QMC algorithms are naturally parallelizable, enabling efficient use of modern 
high-performance computing resources.


%---------------------------------------------------------
\subsubsection*{Mathematical Requirements on the Trial Wavefunction}

To ensure that the variational energy and local energy are finite, the following conditions 
must be satisfied whenever the potential is finite:
\begin{enumerate}
    \item $\Psi_T$ and its gradient $\nabla \Psi_T$ must be continuous.
    \item The integrals $\int \Psi_T^\* \Psi_T\, dR$, 
         $\int \Psi_T^\* H \Psi_T\, dR$, and 
         $\int \Psi_T^\* H^2 \Psi_T\, dR$  
         must exist.
\end{enumerate}

These conditions also guarantee a finite variance of the energy.  
The ``zero-variance property'' of QMC states that if $\Psi_T$ equals an exact eigenstate,  
then the local energy becomes constant and the variance goes to zero,  
dramatically improving Monte Carlo efficiency. These limitations are further discussed in Ref.~\cite{FMNR01}, Sec.~III~C.



%---------------------------------------------------------



%---------------------------------------------------------
\subsubsection*{Structure of a Useful Trial Wavefunction}

A common and highly effective ansatz for electronic systems is the 
\emph{Slater--Jastrow} form:
\[
\Psi_T(R;\alpha) = \Phi(R)\, J(R;\alpha),
\]
where $R = (r_1, \ldots, r_N)$ is the configuration of all electrons.  

\paragraph{Slater determinant (antisymmetric part).}
The Slater determinant
\[
\Phi(R) = 
\det \big[ \phi_i(r_j) \big]
\]
ensures antisymmetry under particle exchange, as required for fermions.  
Each orbital $\phi_i$ is a single-particle wavefunction describing where one electron 
``prefers'' to be in space.

A high-quality Slater determinant should satisfy the following properties:
\begin{itemize}
    \item \textbf{Correct symmetry.}  
    The orbitals must transform appropriately under rotations, translations, reflections, 
    and crystal periodicity (in solids).
    \item \textbf{Orthonormality.}  
    Using orthonormal orbitals keeps the determinant numerically stable and produces 
    well-behaved nodal surfaces.
    \item \textbf{Good nodal surfaces.}  
    The nodal surface is the set of configurations where $\Psi_T(R) = 0$.  
    Electrons can never be found at these nodes.  
    HF orbitals often give qualitatively correct nodal surfaces, typically better than 
    DFT--LDA orbitals for atoms and molecules.  
    Since DMC's fixed-node approximation depends entirely on the nodes of $\Psi_T$, 
    poor nodal structure leads directly to higher errors.
    \item \textbf{Physically reasonable shape.}  
    HF orbitals reflect chemical features such as shell structure, electron localization, 
    hybridization, and bonding patterns.
    \item \textbf{Compactness.}  
    HF already provides a single-determinant approximation.  
    Without it, one would require very large configuration-interaction (CI) expansions, 
    slowing down QMC convergence significantly.
\end{itemize}

These properties and the role of the Slater determinant as the antisymmetric
component of the trial wavefunction are discussed in Ref.~\cite{FMNR01},
Secs.~II~C and IV~B.


\paragraph{Jastrow factor (correlation part).}
The Jastrow factor introduces explicit electron--electron correlations:
\[
J(R;\alpha) = \exp\!\left[ \sum_{i<j} u(r_{ij};\alpha) \right],
\]
where $u(r_{ij})$ is a correlation function depending on the inter-electron distance.  

This factor corrects several deficiencies of Hartree--Fock:
\begin{itemize}
    \item \textbf{Cusp conditions.}  
    The exact electronic wavefunction has cusps when two electrons approach each other.  
    HF lacks these entirely.
    \item \textbf{Correlation holes.}  
    The Jastrow factor accounts for electrons avoiding each other at short range.
    \item \textbf{Short- and long-range dynamical correlation.}  
    This includes van der Waals interactions, which HF cannot describe.
\end{itemize}

Together, the Slater and Jastrow parts produce a wavefunction with both accurate symmetry 
structure (Slater) and realistic correlation behavior (Jastrow).


%---------------------------------------------------------
\subsubsection*{Choosing a Good Trial Wavefunction}

Before optimizing the parameters within $\Psi_T$, one must first decide on a suitable
functional form for the trial wavefunction itself.  
For electronic systems, the construction almost always begins with a set of
single-particle orbitals obtained from a mean-field calculation such as Hartree--Fock (HF)
or, in some cases, density functional theory (DFT).  
Although HF neglects dynamical electron correlation, it provides a high-quality
antisymmetric foundation for VMC because:

\begin{itemize}
    \item HF orbitals obey the correct spatial and spin symmetries of the problem,
    \item they are orthonormal and numerically stable,
    \item they produce nodal surfaces that are qualitatively correct for many atoms and molecules,
    \item and HF already captures the dominant exchange physics arising from antisymmetry.
\end{itemize}

As discussed in Sec.~VII of \cite{FMNR01}, these properties make HF orbitals reliable
building blocks for constructing the Slater determinant in VMC.  
In practice, one first performs an HF (or DFT--LDA) calculation for the system of interest
and extracts the resulting orbitals $\{\phi_i(r)\}$.  
These orbitals form the Slater determinant $\Phi(R)$, which enforces the fermionic symmetry
and defines the nodal surface that strongly influences the accuracy of both VMC and DMC.

However, the HF determinant lacks several key features of the exact ground state.
In particular, it does not incorporate:

\begin{itemize}
    \item the electron--electron cusp conditions,
    \item short-range correlation holes,
    \item long-range dynamical correlation,
    \item or van der Waals interactions.
\end{itemize}

To address these deficiencies, VMC introduces a Jastrow correlation factor
$J(R;\alpha)$, whose functional form is chosen to satisfy the known cusp conditions
and to capture both short- and long-range electron correlation.  
Typical Jastrow forms include one-body (electron--ion) and two-body (electron--electron)
terms, with additional three-body terms used in high-accuracy calculations.

The overall process for selecting a good trial wavefunction therefore consists of:
\begin{enumerate}
    \item performing an HF or DFT calculation to obtain physically meaningful orbitals,
    \item constructing the Slater determinant using these orbitals,
    \item choosing a Jastrow factor with the correct cusp behavior and flexible correlation terms,
    \item and verifying that the resulting Slater--Jastrow form is compact, well-behaved,
          and computationally efficient to evaluate.
\end{enumerate}

This Slater--Jastrow wavefunction serves as the starting point for subsequent parameter
optimization, which refines the correlation terms while preserving the symmetry and nodal
structure inherited from the HF determinant.


%---------------------------------------------------------
\subsubsection*{Optimization of Variational Parameters}

The trial wavefunction contains adjustable parameters $\alpha$, typically in the Jastrow factor 
and occasionally in the orbitals.  
These parameters are optimized by minimizing either
\[
E_V(\alpha)
\quad \text{or} \quad
\mathrm{Var}[E_{\mathrm{loc}}],
\]
where $E_{\mathrm{loc}}$ is the local energy.  

\paragraph{Energy minimization.}
The variational principle ensures $E_V(\alpha) \ge E_0$, so minimizing $E_V$ is a direct way 
to improve the wavefunction.  
However, Monte Carlo estimates of $E_V$ can be noisy, making optimization challenging when 
$\Psi_T$ is far from optimal.

\paragraph{Variance minimization.}
Since the local energy of the exact ground state is constant,
variance minimization provides a stable and physically motivated strategy.  
It is widely used in practice, especially in the early stages of optimization.

\paragraph{TODO: METHOD OF OPTIMIZATION OF PARAMETERS}

