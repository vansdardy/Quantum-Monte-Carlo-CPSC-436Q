\subsection{Selection and Optimization of the Trial Wavefunction}

In this section, we focus on zero-temperature VMC applied to electronic systems, where the wavefunction describes $N$ interacting electrons.   Choosing an accurate trial wavefunction $\Psi_T$ is crucial: it determines not only the  variational energy $E_V$ but also the statistical efficiency of the Monte Carlo sampling. The purpose of $\Psi_T$ is to provide the best possible approximation to the true ground state while remaining mathematically well-behaved and computationally tractable.

%---------------------------------------------------------
\subsubsection{Motivation: Why Good Trial Wavefunctions Matter}

Although VMC is a powerful many-body method, its accuracy depends heavily on the quality of $\Psi_T$. Most ground-state calculations in electronic structure theory instead use less expensive methods such as Hartree--Fock (HF) or Kohn--Sham density functional theory (DFT). These methods scale favorably and are efficient to solve, but they have well-known failures, HF neglects correlation entirely, resulting in limited predictive power. These limitations are further discussed in Ref.~\cite{foulkes_mitas_needs_rajagopal_01}, Sec.~II~C. \\
Quantum Monte Carlo techniques successfully overcome many of these failures. Continuum QMC calculations can already achieve ``chemical accuracy''  ($\sim 1$ kcal/mol $\approx 0.004$ eV per molecule) for small systems and show no fundamental loss of accuracy for larger systems (Ref.~\cite{foulkes_mitas_needs_rajagopal_01}, Sec.~I). The computational cost of fermionic VMC and DMC scales approximately as $N^3$ with the number of electrons $N$, making simulations of moderately large systems feasible.  Furthermore, QMC algorithms are naturally parallelizable, enabling efficient use of modern high-performance computing resources.





%---------------------------------------------------------
\subsubsection{Structure of a Useful Trial Wavefunction}

A common and highly effective ansatz for electronic systems is the 
\emph{Slater--Jastrow} form:
\[
\Psi_T(\mathbf{R};\alpha) = \Phi(\mathbf{R})\, J(\mathbf{R};\alpha),
\]
where $\mathbf{R} = (\mathbf{r}_1, \ldots, \mathbf{r}_N)$ denotes the electron coordinates. This is the standard starting point for VMC of electronic systems because it combines an antisymmetric Slater determinant with a Jastrow factor that captures the electron-electron correlation. 

% \paragraph{Slater determinant (antisymmetric part).}
% The Slater determinant
% \[
% \Phi(R) = 
% \det \big[ \phi_i(r_j) \big]
% \]
\begin{definition}
    The Slater determinant that captures the antisymmetric part of an electron system takes the form of:
    \[
    \Phi(\mathbf{R}) = 
    \det \big[ \phi_i(\mathbf{r}_j) \big]
    \]
\end{definition}
which enforces the required antisymmetry of a fermionic wavefunction. Each orbital $\phi_i$ is the wavefunction of a single electron independently moving in a self-consistent field generated by the other electrons. To give extra clarification, the determinant is taken over a matrix in the form: \begin{equation}
    \det (\mathcal{D}(\mathbf{R})) = \begin{vmatrix}
        \phi_1(\mathbf{r}_1) & \phi_1(\mathbf{r}_2) & \cdots & \phi_1(\mathbf{r}_N) \\
        \phi_2(\mathbf{r}_1) & \phi_2(\mathbf{r}_2) & \cdots & \phi_2(\mathbf{r}_N) \\
        \vdots & \vdots & \ddots & \vdots \\
        \phi_N(\mathbf{r}_1) & \phi_2(\mathbf{r}_2) & \cdots & \phi_N(\mathbf{r}_N)
    \end{vmatrix}
\end{equation}
where $i$ represents the orbital, and $j$ represents the electron. \\
A high-quality Slater determinant within VMC must be built from orbitals that respect the symmetries of the Hamiltonian, remain orthonormal for numerical stability, and generate reasonable nodal surfaces, since the behavior of $\Psi_T$ near its nodes strongly influences the stability and accuracy of the variational estimate.  It is also desirable for the determinant to remain compact so that computing $\Phi(\mathbf{R})$ and its derivatives during the Monte Carlo sampling procedure is computationally efficient. In practice, Hartree--Fock orbitals naturally satisfy these conditions and therefore form a reliable foundation for constructing the Slater determinant \cite{foulkes_mitas_needs_rajagopal_01}.




% \paragraph{Jastrow factor (correlation part).}
% The Jastrow factor introduces explicit electron--electron correlation through
% \[
% J(R;\alpha) = \exp\!\left[ \sum_{i<j} u(r_{ij};\alpha) \right],
% \]
\begin{definition}
    The Jastrow factor explicitly introduces electron-electron correlation through
    \[
    J(\mathbf{R};\alpha) = \exp\!\left[ \sum_{i<j} u(\mathbf{r}_{ij};\alpha) \right],
    \]
\end{definition}
where $u(\mathbf{r}_{ij};\alpha)$ is a function of the distance between two electrons, with a set of tuning parameters~$\alpha$ that control how strongly the electrons correlate with one another. \\
The Jastrow term enforces the correct electron--electron cusp behaviour, accounts for short-range correlation holes, and captures long-range effects such as dispersion that Hartree--Fock fails to describe. By including these Coulomb-driven correlation effects, the Jastrow factor lowers the variational energy and reduces fluctuations in the local energy while leaving the antisymmetry of the Slater determinant unchanged. \\
The choice of $u(\mathbf{r}_{ij};\alpha)$ depends on the physical system and on the types of particle-particle correlations that must be captured. Simple analytic forms that satisfy the electron--electron cusp condition are often sufficient for small atoms, while more flexible parametrizations are used for larger or more strongly correlated systems. \\
In our VMC simulation of helium (see \cref{app:vmc_code}), we select a two-body Jastrow factor whose initial parameters follow directly from the cusp condition and from basic physical intuition about the typical separation between the two electrons. In \cref{optimization}, we describe how these parameters are optimized by minimizing the variance of the local energy. 





%---------------------------------------------------------
\subsubsection{Selecting the Slater--Jastrow wavefunction.}

Constructing a good trial wavefunction therefore involves:  
(1) performing a Hartree--Fock (or DFT) calculation to generate orbitals with the appropriate symmetry and nodal structure;  
(2) building the Slater determinant from these orbitals; and  
(3) choosing a Jastrow factor with the correct cusp conditions and flexible correlation terms. The resulting Slater--Jastrow wavefunction provides a balanced and computationally efficient starting point for subsequent parameter optimization within VMC.


%---------------------------------------------------------
\subsubsection{Optimization of Variational Parameters}\label{optimization}

The trial wavefunction $\Psi_T(\mathbf{R};\alpha)$ contains adjustable parameters~$\alpha$, primarily in the Jastrow factor, that determine the strength and range of the correlation terms. To obtain the most accurate and statistically efficient wavefunction, these parameters must be optimized.  In this work we focus on \emph{variance minimization}, which adjusts $\alpha$ so that the fluctuations of the local energy,
\[
E_{\mathrm{loc}}(\mathbf{R};\alpha) = \frac{\hamiltonian\Psi_T(\mathbf{R};\alpha)}{\Psi_T(\mathbf{R};\alpha)},
\]
are minimized over configurations sampled from $|\Psi_T|^2$. For the exact ground state the local energy is constant, so a lower variance directly indicates a trial wavefunction that is closer to the true eigenstate. Variance minimization is therefore a stable and widely used approach. \\
In practice, the optimization proceeds by estimating the derivatives of the variance with respect to each parameter~$\alpha_k$ using configurations drawnfrom an initial Monte Carlo run.  These derivatives provide a stochastic gradient that indicates how the parameters should be updated to reduce the variance.  Repeating this process iteratively yields progressively improved values of~$\alpha$.  The specific update rule used in our implementation is described in \cref{app:param_update}, where we apply this procedure to optimize the Jastrow parameters for the helium VMC calculation.