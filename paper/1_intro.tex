Quantum Monte Carlo (QMC) method is a class of classical computational algorithm that leverages the Monte Carlo method to simulate and approximate the ground-state and thermodynamic equilibria of many-body physical systems. Its essence lies within the fact that one does not need to precisely understand how high-dimensional wavefunctions behave as long as it is possible to sample such wavefunctions so that an approximation to the density distribution is accessible. \\
In Section 2, we introduce the Monte Carlo method through a classic example of estimating $\pi$ and summarizing its core idea. In Section 3, we highlight how Variational Monte Carlo can be used to estimate such ground-state energies, by introducing relevant quantum physical concepts, and explaining the complete scope of steps in detail, while providing a code sample that calculates the electronic ground-state energy of a fixed-nucleus helium atom using VMC. In Section 4 and 5, We explore the other two most-used QMC methods, and highlight key differences among them. Finally, in Section 6, we discuss briefly about how the fermion sign problem affects QMC methods, and some conceptual ways to circumvent the problem.