\appendix
\section{Monte Carlo Method}
\subsection{Using Monte Carlo Method to Find \texorpdfstring{$\pi$}{pi}}\label{app:pi}
The link to the Jupyter Notebook that demonstrates this process can be accessed \href{https://github.com/vansdardy/Quantum-Monte-Carlo-CPSC-436Q/blob/fa0962f7f55efa86513254f068241f26fbd05df1/code/monte-carlo-demo/pi.ipynb}{\textcolor{blue}{here}} on GitHub.

\section{Variational Monte Carlo}

\subsection{Using VMC to Find the Ground-state Energy of the Two-electron System in a Fixed-Nucleus Helium Atom}\label{app:vmc_code}
The link to the Jupyter Notebook that demonstrates this process can be accessed (\dots) on GitHub.

\subsubsection*{Slater Determinant}

For the ground state of helium both electrons occupy the same spatial $1s$
orbital with opposite spins.  The spatial part of the Slater determinant
therefore reduces to a product:
\[
\Phi(R) = \phi_{1s}(r_1)\,\phi_{1s}(r_2),
\qquad
\phi_{1s}(r) \propto e^{-Zr}.
\]
Since normalization constants cancel in Metropolis ratios and in the local
energy, the implementation uses
\[
\ln \Phi(R) = -Z(|r_1| + |r_2|).
\]
\\
This form satisfies the exact electron--nucleus cusp condition and gives an
analytically simple expression for gradients and Laplacians, which reduces the
variance of the local energy estimator.

\subsubsection*{Jastrow Factor}

To incorporate electron--electron correlation we adopt a simple two-body
Jastrow factor of the form
\[
J(R;\beta) = \exp\!\big[u(r_{12};\beta)\big], \qquad
u(r_{12};\beta) = 
\frac{a_{\mathrm{cusp}}\, r_{12}}{1 + \beta r_{12}},
\]
where  
- $r_{12} = |r_1 - r_2|$,  
- $a_{\mathrm{cusp}} = \tfrac{1}{2}$ is fixed by the electron--electron cusp
  condition for opposite-spin electrons,  
- $\beta$ is a variational parameter controlling the range of the correlation hole. \\

The code therefore uses
\[
\ln J(R;\beta) = u(r_{12};\beta),
\qquad
\ln \Psi_T = \ln \Phi + \ln J.
\]

This Jastrow is real-valued and preserves the antisymmetry provided by the Slater determinant.

\subsubsection*{Analytic Local Energy Evaluation}

The local energy is computed using the standard expression
\[
E_{\mathrm{loc}}(R;\beta)
= -\frac{1}{2}
\sum_{i=1}^2
\left[
\nabla_i^2 \ln \Psi_T +
\bigl|\nabla_i \ln \Psi_T\bigr|^2
\right]
- Z 
\left(\frac{1}{|r_1|} + \frac{1}{|r_2|}\right)
+\frac{1}{|r_1 - r_2|}.
\]

To reduce variance, the derivatives of $\ln \Psi_T$ are computed analytically.
For the Slater part,
\[
\nabla_1 \ln \Phi = -Z\,\hat r_1, \qquad
\nabla_2 \ln \Phi = -Z\,\hat r_2,
\]
\[
\nabla_1^2 \ln \Phi = -\frac{2Z}{|r_1|}, \qquad
\nabla_2^2 \ln \Phi = -\frac{2Z}{|r_2|}.
\]

For the Jastrow term ($r = r_{12}$):
\[
u'(r)  = \frac{a_{\mathrm{cusp}}}{(1+\beta r)^2}, \qquad
u''(r) = -\frac{2a_{\mathrm{cusp}}\beta}{(1+\beta r)^3},
\]
\[
\nabla_1 u = u'(r_{12}) \hat s, \qquad
\nabla_2 u = -u'(r_{12}) \hat s,
\]
\[
\nabla_1^2 u = \nabla_2^2 u
= u''(r_{12}) + \frac{2}{r_{12}}u'(r_{12}),
\]
where $s = r_1 - r_2$ and $\hat s = s/|s|$.
\\
The kinetic energy is formed from these analytic derivatives, while the
potential energy uses the standard electron--nucleus and electron--electron
Coulomb interactions.

\subsubsection*{Metropolis Sampling of $|\Psi_T|^2$}

The probability density $|\Psi_T(R;\beta)|^2$ is sampled using the Metropolis
algorithm:\\1. start from a random initial configuration drawn from a Gaussian distribution;  
2. propose $R' = R + \delta R$ with $\delta R$ drawn from a normal distribution
   with width \texttt{step\_size};  
3. accept the move with probability  
\[
A = \min\!\left[
1,\;
\frac{|\Psi_T(R';\beta)|^2}{|\Psi_T(R;\beta)|^2}
\right];
\]
4. after a thermalization period the local energy is recorded at each accepted configuration.\\
The function \texttt{metropolis\_vmc} implements this procedure and returns the
mean local energy, its variance, and the full set of sampled values.

\subsubsection*{Variance-Based Optimization of the Jastrow Parameter}

We optimize the Jastrow parameter $\beta$ by minimizing the variance of the local energy. Since helium has only one variational parameter, a simple scan over a discrete grid,
\[
\beta \in \{\beta_1,\beta_2,\ldots,\beta_M\},
\]
is sufficient.  
The function \texttt{optimize\_beta} evaluates the VMC variance at each value
and returns the $\beta$ which minimizes $\mathrm{Var}(E_{\mathrm{loc}})$.


\subsection{Rule to Update the Parameter in VMC}\label{app:param_update}

\newpage
