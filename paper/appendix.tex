\appendix
\section{Monte Carlo Method}
\subsection{Using Monte Carlo Method to Find \texorpdfstring{$\pi$}{pi}}\label{app:pi}
The link to the Jupyter Notebook that demonstrates this process can be accessed \href{https://github.com/vansdardy/Quantum-Monte-Carlo-CPSC-436Q/blob/fa0962f7f55efa86513254f068241f26fbd05df1/code/monte-carlo-demo/pi.ipynb}{\textcolor{blue}{here}} on GitHub.

\section{Variational Monte Carlo}

\subsection{Using VMC to Find the Ground-state Energy of the Two-electron System in a Fixed-Nucleus Helium Atom}\label{app:vmc_code}
The link to the Jupyter Notebook that demonstrates this process can be accessed (\dots) on GitHub.

\subsection{Rule to Update the Parameter in VMC}\label{app:param_update}
\subsubsection{Slater Determinant}

For the ground state of helium both electrons occupy the same spatial $1s$
orbital with opposite spins.  The spatial part of the Slater determinant
therefore reduces to a product:
\[
\Phi(\mathbf{R}) = \phi_{1s}(\mathbf{r}_1)\,\phi_{1s}(\mathbf{r}_2),
\qquad
\phi_{1s}(\mathbf{r}) \propto e^{-Z\mathbf{r}}.
\]
Since normalization constants cancel in Metropolis ratios and in the local
energy, the implementation uses
\[
\ln \Phi(\mathbf{R}) = -Z(|\mathbf{r}_1| + |\mathbf{r}_2|).
\]
\\
This form satisfies the exact electron--nucleus cusp condition and gives an analytically simple expression for gradients and Laplacians, which reduces the variance of the local energy estimator.

\subsubsection{Jastrow Factor}

To incorporate electron--electron correlation we adopt a simple two-body Jastrow factor of the form
\[
J(\mathbf{R};\beta) = \exp\!\big[u(\mathbf{r}_{12};\beta)\big], \qquad
u(\mathbf{r}_{12};\beta) = \frac{a_{\mathrm{cusp}}\mathbf{r}_{12}}{1 + \beta \mathbf{r}_{12}}
\]
where $\mathbf{r}_{12} = |\mathbf{r}_1 - \mathbf{r}_2|$, $a_{\mathrm{cusp}} = \tfrac{1}{2}$ is fixed by the electron--electron cusp condition for opposite-spin electrons, $\beta$ is a variational parameter controlling the range of the correlation hole. \\
The code therefore uses
\[
\ln J(\mathbf{R};\beta) = u(\mathbf{r}_{12};\beta),
\qquad
\ln \Psi_T = \ln \Phi + \ln J.
\]

This Jastrow is real-valued and preserves the antisymmetry provided by the Slater determinant.

\subsubsection{Analytic Local Energy Evaluation}

The local energy is computed using the standard expression
\[
E_{\mathrm{loc}}(\mathbf{R};\beta) = -\frac{1}{2} \sum_{i=1}^2
\left[
\nabla_i^2 \ln \Psi_T +
\bigl|\nabla_i \ln \Psi_T\bigr|^2
\right]
- Z 
\left(\frac{1}{|\mathbf{r}_1|} + \frac{1}{|\mathbf{r}_2|}\right)
+\frac{1}{|\mathbf{r}_1 - \mathbf{r}_2|}.
\]

To reduce variance, the derivatives of $\ln \Psi_T$ are computed analytically. For the Slater part,
\[
\nabla_1 \ln \Phi = -Z\,\hat{\mathbf{r}}_1, \qquad
\nabla_2 \ln \Phi = -Z\,\hat{\mathbf{r}}_2,
\]
\[
\nabla_1^2 \ln \Phi = -\frac{2Z}{|\mathbf{r}_1|}, \qquad
\nabla_2^2 \ln \Phi = -\frac{2Z}{|\mathbf{r}_2|}.
\]

For the Jastrow term ($\mathbf{r} = \mathbf{r}_{12}$):
\[
u'(\mathbf{r})  = \frac{a_{\mathrm{cusp}}}{(1+\beta \mathbf{r})^2}, \qquad
u''(r) = -\frac{2a_{\mathrm{cusp}}\beta}{(1+\beta \mathbf{r})^3},
\]
\[
\nabla_1 u = u'(\mathbf{r}_{12}) \hat s, \qquad
\nabla_2 u = -u'(\mathbf{r}_{12}) \hat s,
\]
\[
\nabla_1^2 u = \nabla_2^2 u
= u''(\mathbf{r}_{12}) + \frac{2}{\mathbf{r}_{12}}u'(\mathbf{r}_{12}),
\]
where $s = \mathbf{r}_1 - \mathbf{r}_2$ and $\hat s = s/|s|$.
\\
The kinetic energy is formed from these analytic derivatives, while the potential energy uses the standard electron--nucleus and electron--electron Coulomb interactions.

\subsubsection{Metropolis Sampling of \texorpdfstring{$|\Psi_T|^2$}{|PsiT2|}}

The probability density $|\Psi_T(\mathbf{R};\beta)|^2$ is sampled using the Metropolis algorithm: \\
1. start from a random initial configuration drawn from a Gaussian distribution; \\
2. propose $\mathbf{R}' = \mathbf{R} + \delta \mathbf{R}$ with $\delta \mathbf{R}$ drawn from a normal distribution with width \texttt{step\_size}; \\
3. accept the move with probability
\[
A = \min\!\left[
1,\;
\frac{|\Psi_T(\mathbf{R}';\beta)|^2}{|\Psi_T(\mathbf{R};\beta)|^2}
\right];
\]
4. after a thermalization period the local energy is recorded at each accepted configuration. \\
The function \texttt{metropolis\_vmc} implements this procedure and returns the mean local energy, its variance, and the full set of sampled values.

\subsubsection{Variance-Based Optimization of the Jastrow Parameter}

We optimize the Jastrow parameter $\beta$ by minimizing the variance of the local energy. Since helium has only one variational parameter, a simple scan over a discrete grid,
\[
\beta \in \{\beta_1,\beta_2,\ldots,\beta_M\},
\]
is sufficient. The function \texttt{optimize\_beta} evaluates the VMC variance at each value and returns the $\beta$ which minimizes $\mathrm{Var}(E_{\mathrm{loc}})$.

\newpage
