\appendix
\section{Monte Carlo Method}
\subsection{Using Monte Carlo Method to Find \texorpdfstring{$\pi$}{pi}}\label{app:pi}
The link to the Jupyter Notebook that demonstrates this process can be accessed \href{https://github.com/vansdardy/Quantum-Monte-Carlo-CPSC-436Q/blob/fa0962f7f55efa86513254f068241f26fbd05df1/code/monte-carlo-demo/pi.ipynb}{\textcolor{blue}{here}} on GitHub.

\section{Variational Monte Carlo}

\subsection{Using VMC to Find the Ground-state Energy of the Two-electron System in a Fixed-Nucleus Helium Atom}\label{app:vmc_code}

The accompanying Jupyter Notebook (access \href{https://github.com/vansdardy/Quantum-Monte-Carlo-CPSC-436Q/blob/345e18358732d319059c910e0ee76caa7a9ab406/code/vmc-demo/vmc.ipynb}{\textcolor{blue}{here}}) contains a full implementation of a Variational Monte Carlo (VMC) simulation for the ground-state electronic energy of the helium atom in the Born--Oppenheimer approximation. We treat the nucleus as fixed at the origin and consider only the two interacting electrons. All calculations are performed in atomic units.

\subsection{Variance-Based Optimization of the Jastrow Parameter}\label{app:param_update}
We optimize the Jastrow parameter $\beta$ by minimizing the variance of the local energy. Since helium has only one variational parameter, a simple scan over a discrete grid,
\[
\beta \in \{\beta_1, \beta_2, \ldots, \beta_M\}
\]
For each $\beta$ in this grid, the function \texttt{optimize\_beta} runs a short VMC simulation, computes the corresponding variance of $E_{\mathrm{loc}}$, and identifies the value of $\beta$ that yields the smallest variance. This optimized parameter is then used in a longer VMC run to produce the final energy estimate.

\newpage
