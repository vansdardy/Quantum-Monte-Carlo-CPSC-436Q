\appendix
\section{Monte Carlo Method}
\subsection{Using Monte Carlo Method to Find \texorpdfstring{$\pi$}{pi}}\label{app:pi}
The link to the Jupyter Notebook that demonstrates this process can be accessed \href{https://github.com/vansdardy/Quantum-Monte-Carlo-CPSC-436Q/blob/fa0962f7f55efa86513254f068241f26fbd05df1/code/monte-carlo-demo/pi.ipynb}{\textcolor{blue}{here}} on GitHub.

\section{Variational Monte Carlo}

\subsection{Using VMC to Find the Ground-state Energy of the Two-electron System in a Fixed-Nucleus Helium Atom}\label{app:vmc_code}

The accompanying Jupyter Notebook (\dots) contains a full implementation of a Variational Monte Carlo (VMC) simulation for the ground-state electronic energy of the helium atom in the Born--Oppenheimer approximation. We treat the nucleus as fixed at the origin and consider only the two interacting electrons. All calculations are performed in atomic units.\\

\subsubsection{Slater Determinant}

For the ground state of helium both electrons occupy the same spatial $1s$ orbital with opposite spins. The spatial part of the Slater determinant therefore reduces to a product:
\[
\Phi(\mathbf{R}) = \phi_{1s}(\mathbf{r}_1)\,\phi_{1s}(\mathbf{r}_2),
\qquad
\phi_{1s}(\mathbf{r}) \propto e^{-Z|\mathbf{r}|}.
\]
Since normalization constants cancel in Metropolis ratios and in the local energy, the implementation uses
\[
\ln \Phi(\mathbf{R}) = -Z(|\mathbf{r}_1| + |\mathbf{r}_2|).
\]
This form satisfies the exact electron--nucleus cusp condition and gives an analytically simple expression for gradients and Laplacians, which reduces the variance of the local energy estimator.\\

\subsubsection{Jastrow Factor}

To incorporate electron--electron correlation we adopt a simple two-body Jastrow factor of the form
\[
J(\mathbf{R};\beta) = \exp\!\big[u(r_{12};\beta)\big], \qquad
u(r_{12};\beta) = \frac{a_{\mathrm{cusp}}\, r_{12}}{1 + \beta r_{12}},
\]
where $r_{12} = |\mathbf{r}_1 - \mathbf{r}_2|$, $a_{\mathrm{cusp}} = \tfrac{1}{2}$ is fixed by the electron--electron cusp condition for opposite-spin electrons, and $\beta$ is a variational parameter controlling the range of the correlation hole. The code therefore uses
\[
\ln J(\mathbf{R};\beta) = u(r_{12};\beta),
\qquad
\ln \Psi_T = \ln \Phi + \ln J.
\]
This Jastrow is real-valued and preserves the antisymmetry provided by the Slater determinant.\\

\subsubsection{Analytic Local Energy Evaluation}

The local energy is computed using the standard expression
\[
E_{\mathrm{loc}}(\mathbf{R};\beta) = -\frac{1}{2} \sum_{i=1}^2
\left[
\nabla_i^2 \ln \Psi_T +
\bigl|\nabla_i \ln \Psi_T\bigr|^2
\right]
- Z 
\left(\frac{1}{|\mathbf{r}_1|} + \frac{1}{|\mathbf{r}_2|}\right)
+ \frac{1}{|\mathbf{r}_1 - \mathbf{r}_2|}.
\]
To reduce variance, the derivatives of $\ln \Psi_T$ are computed analytically. For the Slater part,
\[
\nabla_1 \ln \Phi = -Z\,\hat{\mathbf{r}}_1, \qquad
\nabla_2 \ln \Phi = -Z\,\hat{\mathbf{r}}_2,
\]
\[
\nabla_1^2 \ln \Phi = -\frac{2Z}{|\mathbf{r}_1|}, \qquad
\nabla_2^2 \ln \Phi = -\frac{2Z}{|\mathbf{r}_2|}.
\]
\noindent
For the Jastrow term (with $r = r_{12}$),
\[
u'(r) = \frac{a_{\mathrm{cusp}}}{(1+\beta r)^2}, \qquad
u''(r) = -\frac{2a_{\mathrm{cusp}}\beta}{(1+\beta r)^3},
\]
\[
\nabla_1 u = u'(r_{12})\, \hat{s}, \qquad
\nabla_2 u = -u'(r_{12})\, \hat{s},
\]
\[
\nabla_1^2 u = \nabla_2^2 u
= u''(r_{12}) + \frac{2}{r_{12}}\, u'(r_{12}),
\]
where $s = \mathbf{r}_1 - \mathbf{r}_2$ and $\hat{s} = s/|s|$. The kinetic energy is assembled from these analytic derivatives, while the potential energy uses the standard Coulomb interactions.\\

\subsubsection{Metropolis Sampling of \texorpdfstring{$|\Psi_T|^2$}{|PsiT2|}}

Configurations are sampled from $|\Psi_T(\mathbf{R};\beta)|^2$ using a Metropolis random walk with a \emph{uniform} proposal distribution. The sampling procedure is:
\begin{enumerate}
\item Initialize the configuration deterministically at
\[
\mathbf{R}_0 =
\bigl( (0.5, 0, 0),\, (0, 0.5, 0) \bigr),
\]
which lies near the region of highest probability density of the $1s$ orbital. The first configuration is immediately stored.\\

\item Propose a trial move
\[
\mathbf{R}' = \mathbf{R} + \delta\mathbf{R},
\]
where each component of $\delta\mathbf{R}$ is drawn uniformly from
\[
\delta\mathbf{R}_{ij} \sim
\mathrm{Uniform}\!\left(-\frac{\texttt{step\_size}}{2},
+\frac{\texttt{step\_size}}{2}\right).
\]\\

\item Accept the move with probability
\[
A = \min\!\left(1,\;
\frac{\Psi_T(\mathbf{R}';\beta)^2}
     {\Psi_T(\mathbf{R};\beta)^2}\right).
\]\\

\item Whether accepted or rejected, the resulting configuration is appended directly to the sample list.\\
\end{enumerate}
\noindent
This procedure generates a Markov chain of electron configurations distributed according to $|\Psi_T|^2$.\\

\subsubsection{Variance-Based Optimization of the Jastrow Parameter}

The Jastrow parameter $\beta$ is optimized by minimizing the estimated variance of the local energy. Since the helium trial wavefunction contains only a single variational parameter, we perform a simple grid search over a small set of candidate values,
\[
\beta \in \{\beta_1, \beta_2, \ldots, \beta_M\}.
\]
For each $\beta$ in this grid, the function \texttt{optimize\_beta} runs a short VMC simulation, computes the corresponding variance of $E_{\mathrm{loc}}$, and identifies the value of $\beta$ that yields the smallest variance. This optimized parameter is then used in a longer VMC run to produce the final energy estimate.\\

\newpage
