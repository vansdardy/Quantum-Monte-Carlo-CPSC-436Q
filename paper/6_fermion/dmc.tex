\subsection{DMC - Fixed-node Approximation}\label{dmc_sign}
The principle that DMC uses to ``circumvent'' the sign problem is to force the ground state of a fermionic system, described by a wavefunction $\Psi_F$, to have the same nodal structure as the trial wavefunction, $\Psi_T$ \cite{troyer_11}. Specifically, given a trial wavefunction $\Psi_T$ that describes a $N$-particle system, find a trial nodal surface to be a $(3N-1)$-dimensional surface on which the function $\Psi_T$ evaluates to $0$, and crossing such a surface would change the sign of $\Psi_T$ \cite{foulkes_mitas_needs_rajagopal_01}. For a DMC simulation to proceed, this nodal constraint forces the ``walkers'' not to cross the nodal surface described above \cite{troyer_11}. This further indicates that the accuracy would be limited, and the result from such a modification only provides an upper bound to the actual ground-state energy \cite{troyer_11}.