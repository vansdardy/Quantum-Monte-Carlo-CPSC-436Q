\documentclass[11pt]{article}

\usepackage[margin=1in]{geometry}
\usepackage{amsmath,amsfonts,amssymb,amsthm}
\usepackage{braket}

\usepackage[dvipsnames]{xcolor}
\usepackage[colorlinks, linkcolor=OliveGreen, citecolor=purple,urlcolor=purple, bookmarks=true,backref=page]{hyperref}
\usepackage[capitalise, nameinlink]{cleveref}
\usepackage{physics}
\usepackage{graphicx}
% redefine formatting of backwards references
\renewcommand{\backref}[1]{}
\renewcommand{\backreftwosep}{, }
\renewcommand{\backreflastsep}{, }
\renewcommand{\backrefalt}[4]{%
\ifcase #1 %
%
\or
[p.\ #2]%
\else
[pp.\ #2]%
\fi}

% quantum commands
% \newcommand{\ketbra}[2]{|#1\rangle\langle#2|}
\newcommand{\ketbrasame}[1]{|#1\rangle\langle#1|}

% naming commands
\newcommand{\GHZ}{\mathrm{GHZ}}
\newcommand{\EPR}{\mathrm{EPR}}
\input{math_env.tex}

\begin{document}

%%%%%%%%%%%%%%%%%%%%%%%%%%%%%%%%%%%%%%%%%%%%%%%%%%%%%%%%%%%%%%%%%%%%%%%%%%%%%%
\title{Quantum Monte Carlo}

\author{Hannah Baek, Tobias Tian \\
University of British Columbia}

\date{\vspace{-2.5ex} December 8th, 2025}
\maketitle

%%%%%%%%%%%%%%%%%%%%%%%%%%%%%%%%%%%%%%%%%%%%%%%%%%%%%%%%%%%%%%%%%%%%%%%%%%%%%%
\begin{abstract}
This paper explores the practicality and applicability of the various Quantum Monte Carlo methods. This report starts by introducing the key properties and ideas of Monte Carlo methods. It then progresses on how Variational Monte Carlo methods approximate ground-state energies, with a focus on how trial wavefunctions are selected and optimized, while highlighting the sampling algorithm that captures the essence of Monte Carlo. Furthermore, it discusses two more QMC methods, Diffusion Monte Carlo, and Path Integral Monte Carlo, and how they can be leveraged to approximate physical equilibria as well. Finally, this paper briefly highlights how the latter two methods suffer from a common fermion sign problem, and how the discussed QMC methods compare and contrast.
\end{abstract}

%%%%%%%%%%%%%%%%%%%%%%%%%%%%%%%%%%%%%%%%%%%%%%%%%%%%%%%%%%%%%%%%%%%%%%%%%%%%%%
\section*{Next Steps and Updated Timeline}
We highlighted in our proposal that this paper will explore the background of three variants of QMC: PIMC, DMC, and VMC. After generally reading about each of the three variants, we have decided to put our main focus on VMC, where we give explanations as detailed as possible. This is also the one on which we choose to give a code demonstration. Therefore, for the next steps, there are three main goals. The first one is to continue to refine the explanation of VMC at each critical step. The second one is to complete writing the code demonstration for VMC. The last one is to explain DMC and PIMC in a less detailed but still complete way. To summarize, what further needs to be done is:
\begin{enumerate}
    \item Refine the VMC process to explain the two remaining challenges.
    \item Complete writing the code demonstration for VMC.
    \item Explain theoretically what DMC does and how it contrasts with VMC, we will focus on \cite{troyer_11} \cite{foulkes_mitas_needs_rajagopal_01} \cite{kalos_whitlock_08}.
    \item Explore PIMC and briefly touch on the Fermion sign problem, here we will focus on \cite{troyer_11} \cite{troyer_wiese_05}.
\end{enumerate}
Our updated timeline: \\
Nov. 12 - Nov. 16: Complete writing about VMC process. Explain as detailed as possible on how trial wavefunction are chosen and optimized, and how sampling is conducted. \\
Nov. 17 - Nov. 23: Complete writing about both DMC and PIMC. Highlight on the comparison between VMC and DMC, and highlight the sign problem in PIMC. \\
Nov. 24 - Nov. 30: Complete the code demonstration for VMC. \\
Dec. 1 - Dec. 7: Finalize and refine the paper by systematically going through it checking for inconsistencies and errors.


%%%%%%%%%%%%%%%%%%%%%%%%%%%%%%%%%%%%%%%%%%%%%%%%%%%%%%%%%%%%%%%%%%%%%%%%%%%%%%
\section{Introduction}
Quantum Monte Carlo (QMC) method is a class of classical computational algorithm that leverages the Monte Carlo method to simulate and approximate the ground-state and thermodynamic equilibria of many-body physical systems. Its essence lies within the fact that one does not need to precisely understand how high-dimensional wavefunctions behave as long as it is possible to sample such wavefunctions so that an approximation to the density distribution is accessible. \\
In Section 2, we introduce the Monte Carlo method through a classic example of estimating $\pi$ and summarizing its core idea. In Section 3, we highlight how Variational Monte Carlo can be used to estimate such ground-state energies, by introducing relevant quantum physical concepts, and explaining the complete scope of steps in detail, while providing a code sample that calculates the electronic ground-state energy of a fixed-nucleus helium atom using VMC. In Section 4 and 5, We explore the other two most-used QMC methods, and highlight key differences among them. Finally, in Section 6, we discuss briefly about how the fermion sign problem affects QMC methods, and some conceptual ways to circumvent the problem.

%%%%%%%%%%%%%%%%%%%%%%%%%%%%%%%%%%%%%%%%%%%%%%%%%%%%%%%%%%%%%%%%%%%%%%%%%%%%%%
\section{Monte Carlo Method}
Monte Carlo method is a class of classical computational algorithm which aims to find a numerical value of interest through iterative random sampling according to a probability distribution. Such method is often applied when analytical solutions are too complex to solve. A simple case in which Monte Carlo method can be used is to approximate the value of $\pi$ with only the knowledge of how to calculate the area of a circle. The process is outlined as such \cite{kalos_whitlock_08}:
\begin{quote}
    1. Consider a square with edge length of $2$, and centered at $(0, 0)$. \\
    2. Then, draw the inscribed circle of such a square which has radius $1$, also centered at $(0, 0)$. \\
    3. \textbf{Uniformly at random} select a point within the square. \\
    4. Calculate the ratio of the number of points within the circle and the total number of points sampled. This should approximate $\frac{\pi}{4}$. \\
    5. Repeat Step 3 and 4 until a certain threshold. Calculate the approximated value of $\pi$ by multiplying the ratio stated in Step 4 by $4$.
\end{quote}
By applying this process, we can obtain the following result (link to code, see \ref{app:pi}):
\begin{center}
    \includegraphics[scale=0.4]{img/pi-100.png}
    \includegraphics[scale=0.4]{img/pi-1000.png}
    \includegraphics[scale=0.4]{img/pi-10000.png}
\end{center}
The key idea from this demonstration is that random sampling can be used to evaluate a definite integral \cite{kalos_whitlock_08}:
$$I = \Int{0}{1}\Int{0}{\sqrt{1 - x^2}}\dd y \dd x$$
From the above demonstration, it is observed that the main goal of Monte Carlo method is to solve a certain mathematical problem with the core process to be iterated random sampling, where each additional iteration gives more accurate results.

%%%%%%%%%%%%%%%%%%%%%%%%%%%%%%%%%%%%%%%%%%%%%%%%%%%%%%%%%%%%%%%%%%%%%%%%%%%%%%
\section{Variational Monte Carlo}
As the number of interacting particles in a given system scale, analytically solving the Schr\"odinger's equation of the entire system while incorporating the interaction of particles within becomes intractable.In this case, the Variational Monte Carlo (abbreviated as VMC) method provides a practical approximation to the ground state energy.
\subsection{Main Process}
With the mathematical setup, it is possible to describe DMC in the following steps \cite{kalos_whitlock_08}: \\
Step 1: Sample the first generation of points $\mathbf{R}'$ according to a trial wavefunction $\Psi_T(\mathbf{R}')$. \\
Step 2: For each point in $\mathbf{R}'$, obtain a new set of points, which form a new configuration $\mathbf{R}$, by sampling according to the Gaussian distribution described by $G_d$, which gives us $\mathbf{R} - \mathbf{R}'$. This is where the idea of Monte Carlo is embedded. \\
Step 3: For each point in $\mathbf{R}$ and its corresponding point in $\mathbf{R}'$, generate $n_d$ according to \cref{eq:walker_clone}. If $n_d = 0$, remove the point from future random walks. If $n_d > 1$, replace the point with $n_d$ clones. \\
Step 4: Adjust $E_T$ accordingly to control the size of ``walkers''. After a sufficient enough time of repeating Step 2 and Step 3, for each point in the final configuration $\mathbf{R}_f$, find the ground-state energy by \cref{eq:approx_energy}.
\subsection{Selection and Optimization of the Trial Wavefunction}

In this section, we focus on zero-temperature VMC applied to electronic systems, where the wavefunction describes $N$ interacting electrons.   Choosing an accurate trial wavefunction $\Psi_T$ is crucial: it determines not only the  variational energy $E_V$ but also the statistical efficiency of the Monte Carlo sampling. The purpose of $\Psi_T$ is to provide the best possible approximation to the true ground state while remaining mathematically well-behaved and computationally tractable.

%---------------------------------------------------------
\subsubsection{Motivation: Why Good Trial Wavefunctions Matter}

Although VMC is a powerful many-body method, its accuracy depends heavily on the quality of $\Psi_T$. Most ground-state calculations in electronic structure theory instead use less expensive methods such as Hartree--Fock (HF) or Kohn--Sham density functional theory (DFT). These methods scale favorably and are efficient to solve, but they have well-known failures, HF neglects correlation entirely, resulting in limited predictive power. These limitations are further discussed in Ref.~\cite{foulkes_mitas_needs_rajagopal_01}, Sec.~II~C. \\
Quantum Monte Carlo techniques successfully overcome many of these failures. Continuum QMC calculations can already achieve ``chemical accuracy''  ($\sim 1$ kcal/mol $\approx 0.004$ eV per molecule) for small systems and show no fundamental loss of accuracy for larger systems (Ref.~\cite{foulkes_mitas_needs_rajagopal_01}, Sec.~I). The computational cost of fermionic VMC and DMC scales approximately as $N^3$ with the number of electrons $N$, making simulations of moderately large systems feasible.  Furthermore, QMC algorithms are naturally parallelizable, enabling efficient use of modern high-performance computing resources.





%---------------------------------------------------------
\subsubsection{Structure of a Useful Trial Wavefunction}

A common and highly effective ansatz for electronic systems is the 
\emph{Slater--Jastrow} form:
\[
\Psi_T(\mathbf{R};\alpha) = \Phi(\mathbf{R})\, J(\mathbf{R};\alpha),
\]
where $\mathbf{R} = (\mathbf{r}_1, \ldots, \mathbf{r}_N)$ denotes the electron coordinates. This is the standard starting point for VMC of electronic systems because it combines an antisymmetric Slater determinant with a Jastrow factor that captures the electron-electron correlation. 

% \paragraph{Slater determinant (antisymmetric part).}
% The Slater determinant
% \[
% \Phi(R) = 
% \det \big[ \phi_i(r_j) \big]
% \]
\begin{definition}
    The Slater determinant that captures the antisymmetric part of an electron system takes the form of:
    \[
    \Phi(\mathbf{R}) = 
    \det \big[ \phi_i(\mathbf{r}_j) \big]
    \]
\end{definition}
which enforces the required antisymmetry of a fermionic wavefunction. Each orbital $\phi_i$ is the wavefunction of a single electron independently moving in a self-consistent field generated by the other electrons. To give extra clarification, the determinant is taken over a matrix in the form: \begin{equation}
    \det (\mathcal{D}(\mathbf{R})) = \begin{vmatrix}
        \phi_1(\mathbf{r}_1) & \phi_1(\mathbf{r}_2) & \cdots & \phi_1(\mathbf{r}_N) \\
        \phi_2(\mathbf{r}_1) & \phi_2(\mathbf{r}_2) & \cdots & \phi_2(\mathbf{r}_N) \\
        \vdots & \vdots & \ddots & \vdots \\
        \phi_N(\mathbf{r}_1) & \phi_2(\mathbf{r}_2) & \cdots & \phi_N(\mathbf{r}_N)
    \end{vmatrix}
\end{equation}
where $i$ represents the orbital, and $j$ represents the electron. \\
A high-quality Slater determinant within VMC must be built from orbitals that respect the symmetries of the Hamiltonian, remain orthonormal for numerical stability, and generate reasonable nodal surfaces, since the behavior of $\Psi_T$ near its nodes strongly influences the stability and accuracy of the variational estimate.  It is also desirable for the determinant to remain compact so that computing $\Phi(\mathbf{R})$ and its derivatives during the Monte Carlo sampling procedure is computationally efficient. In practice, Hartree--Fock orbitals naturally satisfy these conditions and therefore form a reliable foundation for constructing the Slater determinant \cite{foulkes_mitas_needs_rajagopal_01}.




% \paragraph{Jastrow factor (correlation part).}
% The Jastrow factor introduces explicit electron--electron correlation through
% \[
% J(R;\alpha) = \exp\!\left[ \sum_{i<j} u(r_{ij};\alpha) \right],
% \]
\begin{definition}
    The Jastrow factor explicitly introduces electron-electron correlation through
    \[
    J(\mathbf{R};\alpha) = \exp\!\left[ \sum_{i<j} u(\mathbf{r}_{ij};\alpha) \right],
    \]
\end{definition}
where $u(\mathbf{r}_{ij};\alpha)$ is a function of the distance between two electrons, with a set of tuning parameters~$\alpha$ that control how strongly the electrons correlate with one another. \\
The Jastrow term enforces the correct electron--electron cusp behaviour, accounts for short-range correlation holes, and captures long-range effects such as dispersion that Hartree--Fock fails to describe. By including these Coulomb-driven correlation effects, the Jastrow factor lowers the variational energy and reduces fluctuations in the local energy while leaving the antisymmetry of the Slater determinant unchanged. \\
The choice of $u(\mathbf{r}_{ij};\alpha)$ depends on the physical system and on the types of particle-particle correlations that must be captured. Simple analytic forms that satisfy the electron--electron cusp condition are often sufficient for small atoms, while more flexible parametrizations are used for larger or more strongly correlated systems. \\
In our VMC simulation of helium (see \cref{app:vmc_code}), we select a two-body Jastrow factor whose initial parameters follow directly from the cusp condition and from basic physical intuition about the typical separation between the two electrons. In \cref{optimization}, we describe how these parameters are optimized by minimizing the variance of the local energy. 





%---------------------------------------------------------
\subsubsection{Selecting the Slater--Jastrow wavefunction.}

Constructing a good trial wavefunction therefore involves:  
(1) performing a Hartree--Fock (or DFT) calculation to generate orbitals with the appropriate symmetry and nodal structure;  
(2) building the Slater determinant from these orbitals; and  
(3) choosing a Jastrow factor with the correct cusp conditions and flexible correlation terms. The resulting Slater--Jastrow wavefunction provides a balanced and computationally efficient starting point for subsequent parameter optimization within VMC.


%---------------------------------------------------------
\subsubsection{Optimization of Variational Parameters}\label{optimization}

The trial wavefunction $\Psi_T(\mathbf{R};\alpha)$ contains adjustable parameters~$\alpha$, primarily in the Jastrow factor, that determine the strength and range of the correlation terms. To obtain the most accurate and statistically efficient wavefunction, these parameters must be optimized.  In this work we focus on \emph{variance minimization}, which adjusts $\alpha$ so that the fluctuations of the local energy,
\[
E_{\mathrm{loc}}(\mathbf{R};\alpha) = \frac{\hamiltonian\Psi_T(\mathbf{R};\alpha)}{\Psi_T(\mathbf{R};\alpha)},
\]
are minimized over configurations sampled from $|\Psi_T|^2$. For the exact ground state the local energy is constant, so a lower variance directly indicates a trial wavefunction that is closer to the true eigenstate. Variance minimization is therefore a stable and widely used approach. \\
In practice, the optimization proceeds by estimating the derivatives of the variance with respect to each parameter~$\alpha_k$ using configurations drawnfrom an initial Monte Carlo run.  These derivatives provide a stochastic gradient that indicates how the parameters should be updated to reduce the variance.  Repeating this process iteratively yields progressively improved values of~$\alpha$.  The specific update rule used in our implementation is described in \cref{app:param_update}, where we apply this procedure to optimize the Jastrow parameters for the helium VMC calculation.
\subsection{Sampling Method}\label{metropolis}
In the previous section, it is discussed how a trial wavefunction can be selected and optimized. Yet, it is still unclear how a configuration $\mathbf{R}$ is sampled such that it satisfies the probability density in \cref{eq:pdf}. \\
Na\"ively, one should sample according to this probability density function. However, it should be noted that the integral present in the denominator of \cref{eq:pdf} becomes computationally implausible to calculate or normalize when the dimension of $\mathbf{R}$ scales, because as stated before, configuration $\mathbf{R}$ is a $3N$-dimensional vector. Furthermore, uniformly sampling in a $3N$-dimensional space suffers from the ``curse of dimensionality'' where if only a given number of data points are being sampled, the samples would be so sparse that they cannot represent the population well \cite{metropolis_rosenbluth_rosenbluth_teller_teller_53}. Furthermore, depending on the form of the selected trial wavefunction, the wavefunction can take different shapes and correlations which further complicates a na\"ive sampling method. \\
Hence, Metropolis et al. developed an algorithm that accepts or rejects a configuration starting from an initial configuration using only probability ratios \cite{metropolis_rosenbluth_rosenbluth_teller_teller_53}. This algorithm ensures that the final list of accepted configurations follows the same distribution as the probability density distribution \cite{metropolis_rosenbluth_rosenbluth_teller_teller_53}. \\
Analogous to the paper in which Metropolis et al. demonstrated how they sampled according to the Boltzmann distribution, the following steps is taken to sample the selected trial wavefunction \cite{metropolis_rosenbluth_rosenbluth_teller_teller_53}:
\begin{quote}
    1. Place the $N$ particles in any configuration, and denote it as $\mathbf{R}_1$. Calculate $\abs{\Psi_T(\mathbf{R}_1)}^2$. \\
    2. Since the initial configuration is a vector that contains the Cartesian positions of $N$ particles, it is possible to select one particle, say particle $i$, who has the initial position of $(x_i, y_i, z_i)$, and displace particle within a small cube, say with an edge length of $\varepsilon$. Mathematically, this displacement is denoted as:
    $$x_i \mapsto x_i + \varepsilon \delta_x, y_i \mapsto y_i + \varepsilon \delta_y, z_i \mapsto z_i + \varepsilon \delta_z$$
    where $\delta_x, \delta_y, \delta_z \in [-1, 1]$. Denote the new configuration as $\mathbf{R}_2$. Calculate $\abs{\Psi_T(\mathbf{R}_2)}^2$. \\
    3. Calculate the ratio between the amplitude probabilities of the two configurations:
    $$p := \frac{\abs{\Psi_T(\mathbf{R}_2)}^2}{\abs{\Psi_T(\mathbf{R}_1)}^2}$$
    4. If $p \ge 1$, confirm moving the particle to the new position. If $p < 1$, move the particle to the new position with probability $p$. Mathematically, move the particle to the new position with probability
    $$P = \min(1, p)$$
    5. If the new configuration is accepted, repeat step 2 to 4 which generates another configuration and determine whether to accept or reject it based on the latest accepted configuration. If the new configuration is rejected, start from the latest accepted configuration, and repeat step 2 to 4. \\
    6. This above process returns a list of $M$ configurations $\mathbf{R}_1, \mathbf{R}_2, \dots, \mathbf{R}_M$.
\end{quote}
This algorithm developed by Metropolis et al. prevents one from needing to sample configurations that grows exponentially large as the dimension scales, and efficiently generates a set of configurations who satisfy the probability distribution from the trial wavefunction. \\
From the list of $M$ configurations, it is then possible to calculate the local energy for each configuration following \cref{eq:local}, and compute the variational approximation of the ground-state energy using \cref{eq:approx_energy}.

%%%%%%%%%%%%%%%%%%%%%%%%%%%%%%%%%%%%%%%%%%%%%%%%%%%%%%%%%%%%%%%%%%%%%%%%%%%%%%
\section{Diffusion Monte Carlo}
Similar to VMC, the Diffusion Monte Carlo (abbreviated as DMC) method provides an alternative way of approximating the ground-state properties of a many-body system. \\
There exists a key difference between VMC and DMC. VMC method selects a trial wavefunction with respect to certain parameters and updates these parameters from results of an iteration to optimize the wavefunction, which leads to the convergence to ground-state properties. DMC method, however, starts with a trial wavefunction, instead of optimizing this wavefunction through updating parameters, it samples certain ``walkers'', which will be explained in detail later, and allows them to diffuse over time with respect to an appropriate function, such that they converge to the ground state.
\subsection{Mathematical Background}\label{dmc_math}
Additional to \cref{time-dep-defn}, DMC is based on the solution to a ``modified'' version of the time-dependent Schr\"odinger's equation, where it is written in imaginary time \cite{troyer_11} \cite{foulkes_mitas_needs_rajagopal_01} \cite{kalos_whitlock_08} \cite{acioli_97}:
\begin{definition}
    A time-dependent Schr\"odinger's equation written in imaginary time is
    $$-\pdv{\tau} \Psi(\mathbf{R}, \tau) = (\hamiltonian - E_T)\Psi(\mathbf{R}, \tau)$$
    where $\tau = \frac{\imag t}{\hbar}$, and $E_T$ is an energy offset.
\end{definition}
Additionally, in \cite{troyer_11}, it is shown that as $\tau \to \infty$, the solution wavefunction to the above equation will have the ground state dominating other states. Furthermore, the introduction of the energy offset allows the wavefunction to be normalized \cite{troyer_11}. \\
For the wavefunction to ``diffuse'' in time, Green's function is needed to describe the process, where \cite{troyer_11} \cite{foulkes_mitas_needs_rajagopal_01} \cite{kalos_whitlock_08} \cite{acioli_97}:
\begin{definition}
    A Green's function that describes the diffusion process for the wavefunction described above, $\Psi(\mathbf{R}, \tau)$, is given to be:
    \begin{equation}\label{eq:green_fn}
        G(\mathbf{R}, \mathbf{R}', \delta\tau) = \mel{\mathbf{R}}{\exp(-\delta\tau(\hamiltonian - E_T))}{\mathbf{R}'}
    \end{equation}
    where $\delta \tau$ is the change in time.
\end{definition}
This permits us to write the general wavefunction to be the result of a diffusion process \cite{troyer_11} \cite{foulkes_mitas_needs_rajagopal_01} \cite{kalos_whitlock_08} \cite{acioli_97}:
$$\Psi(\mathbf{R}, \tau) = \int G(\mathbf{R}, \mathbf{R}', \tau)\Psi(\mathbf{R}', 0)\dd \mathbf{R}'$$
where the Green's function acts like the probability density that the starting configuration $\mathbf{R}'$ evolves to $\mathbf{R}$ after time $\tau$. \\
As $\delta \tau \to 0$, while assuming that $\hamiltonian = \hat{T} + \hat{V}$, where $\hat{T}$ is the kinetic-energy operator and $\hat{V}$ is the potential-energy operator, by applying the Trotter-Suzuki formula and approximation for small $\delta \tau$, \cref{eq:green_fn} can be rewritten as \cite{troyer_11} \cite{foulkes_mitas_needs_rajagopal_01} \cite{acioli_97}:
\begin{equation}\label{eq:green_approx}
    G(\mathbf{R}, \mathbf{R}', \delta \tau) = (2 \pi \delta\tau)^{-\frac{3N}{2}} \exp(-\frac{(\mathbf{R} - \mathbf{R}')^2}{2\delta\tau})\exp(-\delta\tau\frac{V(\mathbf{R}) + V(\mathbf{R}') - 2E_T}{2})
\end{equation}
where the rate term \cite{kalos_whitlock_08}
\begin{equation}\label{eq:rate_term}
    p := \exp(-\delta\tau\frac{V(\mathbf{R}) + V(\mathbf{R}') - 2E_T}{2})
\end{equation}
determines the number of ``walkers'' that survive to the next step \cite{foulkes_mitas_needs_rajagopal_01}. For each original ``walker'', the number of descendants that ``branches'' out is:
\begin{equation}\label{eq:walker_clone}
    n_d := \text{int}(p + \eta)
\end{equation}
where $\text{int}$ takes the integer part of the value, and $\eta$ is a random number drawn uniformly at random on the interval $[0,1]$ \cite{troyer_11} \cite{foulkes_mitas_needs_rajagopal_01}. \\
Additionally, the rest of \cref{eq:green_approx} excluding \cref{eq:rate_term} is a time propagator, $G_d$, which generates a random walk, and is a Gaussian distribution \cite{troyer_11}\cite{kalos_whitlock_08}.
\subsection{Main Process}
With the mathematical setup, it is possible to describe DMC in the following steps \cite{kalos_whitlock_08}: \\
Step 1: Sample the first generation of points $\mathbf{R}'$ according to a trial wavefunction $\Psi_T(\mathbf{R}')$. \\
Step 2: For each point in $\mathbf{R}'$, obtain a new set of points, which form a new configuration $\mathbf{R}$, by sampling according to the Gaussian distribution described by $G_d$, which gives us $\mathbf{R} - \mathbf{R}'$. This is where the idea of Monte Carlo is embedded. \\
Step 3: For each point in $\mathbf{R}$ and its corresponding point in $\mathbf{R}'$, generate $n_d$ according to \cref{eq:walker_clone}. If $n_d = 0$, remove the point from future random walks. If $n_d > 1$, replace the point with $n_d$ clones. \\
Step 4: Adjust $E_T$ accordingly to control the size of ``walkers''. After a sufficient enough time of repeating Step 2 and Step 3, for each point in the final configuration $\mathbf{R}_f$, find the ground-state energy by \cref{eq:approx_energy}.
\subsection{Importance Sampling}
The process described above is generally extremely inefficient for a large number of particles due to possible complicated inter-particle interactions \cite{troyer_11}, which can lead to \cref{eq:rate_term} fluctuate wildly \cite{foulkes_mitas_needs_rajagopal_01}. The technique of \textit{importance sampling} is introduced to overcome these difficulties. Without extending into the mathematical details embedded in this technique, the key idea of the technique can be described as the following:
\begin{quote}
    1. Design a \textit{trial wavefunction}, $\Phi(\mathbf{R})$, that approximately describes the exact ground state $\Phi_0$ \cite{troyer_11}. \\
    2. Introduce a new function $f(\mathbf{R}, \tau)$ such that \cite{troyer_11} \cite{foulkes_mitas_needs_rajagopal_01} \cite{acioli_97}
    \begin{equation}
        f(\mathbf{R}, \tau) := \Phi(\mathbf{R})\Psi_T(\mathbf{R}, \tau)
    \end{equation}
    where $f(\mathbf{R}, \tau)$ is interpreted as the probability density distribution of the population of ``walkers'' instead of $\Psi_T(\mathbf{R}, \tau)$. \\
    3. Follow a similar diffusion process as described in the main process, except that, before deciding to ``branch'' or not, one need to introduce a \textit{drift}, or \textit{pseudo force}, $\mathbf{F}$, to the new configuration $\mathbf{R}$ relative to $\mathbf{R}'$, where \cite{troyer_11}
    \begin{equation}
        \mathbf{R} = \mathbf{R}' + \frac{\delta \tau}{2} \mathbf{F}(\mathbf{R}')
    \end{equation}
    and
    \begin{equation}
        \mathbf{F} = \frac{2 \nabla\Phi(\mathbf{R})}{\Phi(\mathbf{R})}
    \end{equation}
\end{quote}
The introduction of the ``drift'' guides the walkers in regions with high probability, hence, reducing fluctuations in \cref{eq:rate_term} and local energies \cite{troyer_11}. \\
Another difficulty that DMC runs into is that it requires the wavefunction to be positive definite \cite{foulkes_mitas_needs_rajagopal_01}. This means DMC can simulate Bosonic systems at zero temperature, but for fermionic systems like electronic systems that have positive and negative values due to antisymmetry, DMC runs into the \textit{fermion sign problem}, which will be discussed in \cref{dmc_sign}. Additionally, DMC also needs tuning if the system of interest consists of bosons in excited states \cite{troyer_11}.

%%%%%%%%%%%%%%%%%%%%%%%%%%%%%%%%%%%%%%%%%%%%%%%%%%%%%%%%%%%%%%%%%%%%%%%%%%%%%%
\section{Path Integral Monte Carlo}
Though the name may be deceptive, Path Integral Monte Carlo (abbreviated as PIMC) shares the same kernel as DMC.

%%%%%%%%%%%%%%%%%%%%%%%%%%%%%%%%%%%%%%%%%%%%%%%%%%%%%%%%%%%%%%%%%%%%%%%%%%%%%%
\section*{Acknowledgments}
If you have discussions with your classmates about your chosen topic, you can thank them here.
\newpage

%%%%%%%%%%%%%%%%%%%%%%%%%%%%%%%%%%%%%%%%%%%%%%%%%%%%%%%%%%%%%%%%%%%%%%%%%%%%%%
\bibliography{references}
\bibliographystyle{alphaurl}
\end{document}